\section{Arjun Yuda Firwanda}
\subsection{Soal 1}
Isi jawaban soal ke-1

Kalau mau dibikin paragrap \textbf{cukup enter aja}, tidak usah pakai \verb|par| dsb

%\subsection{Soal 2}
%Isi jawaban soal ke-2

%\subsection{Soal 3}
%Isi jawaban soal ke-3

\section{Dwi Yulianingsih}
\subsection{Soal 1}
Buatlah fungsi (file terpisah/library dengan nama NPM csv.py) untuk membuka file csv dengan lib csv mode list
\lstinputlisting[firstline=10, lastline=20]{src/4/1174009/praktek/d1174009_csv.py}


\subsection{Soal 2}
Buatlah fungsi (file terpisah/library dengan nama NPM csv.py) untuk membuka file csv dengan lib csv mode dictionary
\lstinputlisting[firstline=22,lastline=34]{src/4/1174009/praktek/d1174009_csv.py}

\subsection{Soal 3}
Buatlah fungsi (file terpisah/library dengan nama NPM pandas.py) untuk membuka file csv dengan lib pandas mode list
\lstinputlisting[firstline=7,lastline=10]{src/4/1174009/praktek/d1174009_pandas.py}

\subsection{Soal 4}
Buatlah fungsi (file terpisah/library dengan nama NPM pandas.py) untuk membuka file csv dengan lib pandas mode dictionary
\lstinputlisting[firstline=12,lastline=15]{src/4/1174009/praktek/d1174009_pandas.py}

\subsection{Soal 5}
Buat fungsi baru di NPM pandas.py untuk mengubah format tanggal menjadi standar dataframe
\lstinputlisting[firstline=17,lastline=19]{src/4/1174009/praktek/d1174009_pandas.py}

\subsection{Soal 6}
Buat fungsi baru di NPM pandas.py untuk mengubah index kolom
\lstinputlisting[firstline=21,lastline=23]{src/4/1174009/praktek/d1174009_pandas.py}

\subsection{Soal 7}
Buat fungsi baru di NPM pandas.py untuk mengubah atribut atau nama kolom
\lstinputlisting[firstline=25,lastline=29]{src/4/1174009/praktek/d1174009_pandas.py}

\subsection{Soal 8}
Buat program main.py yang menggunakan library NPM csv.py yang membuat dan membaca file csv
\lstinputlisting[firstline=8,lastline=10]{src/4/1174009/praktek/main_dwi.py}

\subsection{Soal 9}
Buat program main2.py yang menggunakan library NPM pandas.py yang membuat dan membaca file csv
\lstinputlisting[firstline=12,lastline=14]{src/4/1174009/praktek/main_dwi.py}

\subsection{Penanganan eror}
Ada kalanya saat kita baca file, tapi filenya belum ada. Maka biasanya akan terjadi IOerror.
\lstinputlisting[firstline=8,lastline=8]{src/4/1174009/praktek/eror.py}
maka di tangani dengan cara seperti dibawah ini :
\lstinputlisting[firstline=10,lastline=13]{src/4/1174009/praktek/eror.py}
maka akan muncul peringatan seperti dibawah :
\lstinputlisting[firstline=15,lastline=15]{src/4/1174009/praktek/eror.py}



\section{Harun Ar-Rasyid}
\subsection{Soal 1}
Berikut adalah pemanggilan file csv dengan library csv yang menggunakan list
\lstinputlisting[firstline=10, lastline=20]{src/4/1174027/praktek/c_1174027_csv.py}

\subsection{Soal 2}
Berikut adalah pemanggilan file csv dengan library csv yang menggunakan dictionary
\lstinputlisting[firstline=22, lastline=31]{src/4/1174027/praktek/c_1174027_csv.py}

\subsection{Soal 3}
Berikut adalah pemanggilan file csv dengan library pandas yang menggunakan list
\lstinputlisting[firstline=9, lastline=11]{src/4/1174027/praktek/p_1174027_pandas.py}

\subsection{Soal 4}
Berikut adalah pemanggilan file csv dengan library pandas yang menggunakan dictionary
\lstinputlisting[firstline=13, lastline=16]{src/4/1174027/praktek/p_1174027_pandas.py}

\subsection{Soal 5}
Berikut penggunaan untuk merubah standar penulisan tanggal, yang mengikuti standar penulisan dari pandas.
\lstinputlisting[firstline=18, lastline=20]{src/4/1174027/praktek/p_1174027_pandas.py}

\subsection{Soal 6}
Berikut merupakan pergantian index kolom
\lstinputlisting[firstline=22, lastline=24]{src/4/1174027/praktek/p_1174027_pandas.py}

\subsection{Soal 7}
berikut merupakan penggunaan untuk merename atribut yang digunakan, atau merubah nama header 0
\lstinputlisting[firstline=26, lastline=30]{src/4/1174027/praktek/p_1174027_pandas.py}

\subsection{Soal 8}
\lstinputlisting[firstline=8, lastline=10]{src/4/1174027/praktek/main_harun.py}

\subsection{Soal 9}
\lstinputlisting[firstline=11, lastline=14]{src/4/1174027/praktek/main_harun.py}

\subsection{Penanganan Error}
Dalam praktek kali ini alhamdulliha tidak menemukan error

\section{Sri Rahayu}
\subsection{Soal 1}
Isi jawaban soal ke-1

Kalau mau dibikin paragrap \textbf{cukup enter aja}, tidak usah pakai \verb|par| dsb

%\subsection{Soal 2}
%Isi jawaban soal ke-2

%\subsection{Soal 3}
%Isi jawaban soal ke-3

\section{Doli Jonviter}
\subsection{Soal 1}
Isi jawaban soal ke-1

Kalau mau dibikin paragrap \textbf{cukup enter aja}, tidak usah pakai \verb|par| dsb

%\subsection{Soal 2}
%Isi jawaban soal ke-2

%\subsection{Soal 3}
%Isi jawaban soal ke-3

\section{Rahmatul Ridha}
\subsection{Soal 1}
Isi jawaban soal ke-1

Kalau mau dibikin paragrap \textbf{cukup enter aja}, tidak usah pakai \verb|par| dsb

%\subsection{Soal 2}
%Isi jawaban soal ke-2

%\subsection{Soal 3}
%Isi jawaban soal ke-3

\section{Tomy Prawoto}
\subsection{Soal 1}
Isi jawaban soal ke-1

Kalau mau dibikin paragrap \textbf{cukup enter aja}, tidak usah pakai \verb|par| dsb

%\subsection{Soal 2}
%Isi jawaban soal ke-2

%\subsection{Soal 3}
%Isi jawaban soal ke-3
